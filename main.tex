\documentclass[a4paper, 12pt]{article}
\usepackage[utf8]{inputenc}
\usepackage{apacite}
\usepackage{geometry}
\usepackage{setspace}

\geometry{margin=1in}

\title{PSYC 400 Term Paper: Personality}
\author{Evin Tunador}
\date{December 2020}

\begin{document}
{\tiny\maketitle}
\doublespacing

Modern personality research is largely focused around trait theory, although other theoretical vantage points do exist. Trait theory defines personality as being made up of multiple dimensions that individuals vary along, and most attempts at defining these dimensions also incorporate a hierarchical aspect with multiple sub-factors  for each dimension. The "Big Five" or "OCEAN" set of factors is the most thoroughly researched and consists of openness, conscientiousness, extraversion, agreeableness, and neuroticism. Also of note is the "HEXACO" model which incorporates a sixth trait for honesty/humility. Recent evidence makes a better case for the latter model, but the former is very similar and benefits from a larger base of research. For these reasons data on the HEXACO model will be discussed where possible, but the majority of the following discussion will detail the Big Five traits unless otherwise stated. \par

%should i go into detail on definitions of each trait? I think I probably should since it would also give me a good place to talk about the cross-cultural change over lifespan & would put that point as a centerpiece

Of particular interest to this paper is the question of trait theory's potential roots in evolutionary developmental psychology. The simplest interpretation of evolutionary theory posits strict fitness maximization leading to convergence of phenotype within a species. This poses the question of how it is that personality can consistently exhibit so much variation; theoretically traits that lead to better fitness outcomes should be selected for while all else is trimmed out of the population by natural selection. Also of note is variation across the lifespan which requires an explanation in terms of deferred and ontogenetic adaptations. The answers to these questions can be seen upon allowing for a more complex interpretation of natural selection in which phenotypes can exhibit trade-offs, leading to a multitude of fitness maximal values rather than a unique maximum. To be more specific, variation along trait dimensions may exist because particular levels may fit different ecological niches and environments, frequency dependencies from game theory may exist, and finally life-history theory posits the coexistence of two distinct fitness maximal strategies. \par

Variation in personality across the lifespan has been repeatedly shown in both cross-sectional and longitudinal American samples with declines in neuroticism, extraversion and openness and increases in agreeableness and conscientiousness during the transition from adolescence to adulthood \cite{fiveCultures1999}. 
\citeauthor{fiveCultures1999} also found a cross-cultural sample of five other countries to concur with the results in American research. This commonality shows that selective pressures on personality differs by age the same way across cultures---likely attributable to the shared challenges of parenthood---which provides evidence for some personality variation as a set of ontogenetic adaptations. \par

Along the lines of trade-offs, variation can be created even within a single species thanks to both numerous niches within a single environment as well as said species frequently having to call multiple environments home. For the former, one might reference jobs that require higher or lower levels of extraversion or how social animals will act differently based on rank in a dominance hierarchy \cite{rainbow2011, lobster1997}. The latter point is especially applicable to humans in reference to our social environment since culture plays a large selective role on personality \cite{cultureClash2006}. These factors can encourage a kind of "built-in" variation in personality since there is no way to predict what niche or environment an organism will exist in at the time of conception. This is also an explanation for why siblings differ so much in personality rather than a parent being genetically forced to put all their eggs into one "personality basket." \par

The next explanation for why trade-offs in personality exist comes from evolutionary game theory. To put it simply, when more than one strategy---in this case a personality trait---exists, the fitness of each might depend upon relative prevalence within the population \cite{gameTheory1999}. An example would be antisocial personality disorder: If a large portion of the population consists of psychopaths, they will inevitably harm each other, leaving the few non-psychopaths in an advantageous position. However if they are relatively rare, then there will exist a plethora of good people to exploit, leaving psychopathy as a viable strategy only in the case that their share of the population is low. Generalizing out towards other personality traits shows how these dynamics can encourage ample variation rather than convergence. \par

The final cause of trade-offs stems from life history theory which posits that species have two possible broad strategies, and that some species including our own can select from the pair at the level of the individual. The "fast" approach coincides with shorter lifespans and involves maximizing the sheer quantity of offspring with relatively little post-copulation investment. In contrast, the "slow" strategy resulting from longer lifespans necessitates fewer offspring with more and higher quality resources allocated towards their survival. An example from humans would be how low honesty facilitates fertility in the short term while the reverse helps with long term reproduction as measured by number of children and grandchildren respectively \cite{personalityFitness2018}. Some researchers have also argued that the Big Five can be further subsumed by one or two higher order factors that may reflect variability in life history strategy \cite{bigOne2007, bigTwo2006}. \par

While construing phenotypes as having fitness pros and cons is the best supported theory, other potential explanations do exist. One is that of selective neutrality, which states that high levels of variation persist because no value within the witnessed range leads to either advantage nor disadvantage\cite{hexaco2016}. However, the assumption of personality traits being irrelevant to natural selection does not hold at least for the modern era, as \citeA{personalityFitness2018} have found benefits of certain personality traits on number of children and grandchildren. The other, called mutation-selection balance, involves high rates of mutation offsetting the population trimming resulting from selection pressures\cite{hexaco2016}. This hypothesis can hold even under the findings of \citeauthor{personalityFitness2018}, but it is in need of more support before it can be considered a serious contender. Overall, each personality trait exhibiting trade-offs with fitness values stemming from a combination of environment, niche, frequency dependencies, and chosen strategy from life-history theory is the most plausible explanation for the diverse personality types present among humans. 

%organisms are adaptation executors rather than fitness maximizers; it's only at the population level that fitness is maximized

\newpage
\bibliographystyle{apacite}
\bibliography{bib}
\end{document}
